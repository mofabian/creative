
\documentclass{article}
\title{Civil violence, agent based model}
\author{Fabian M\"onkeberg}
\date{\today}
\usepackage[ngerman]{babel}
\usepackage{fancyhdr}
\usepackage{amsmath}
\pagestyle{fancy}
\renewcommand{\sectionmark}[1]{\markboth{\emph{#1}}{}}
\fancyhf{}
\fancyhead[LE,RO]{\textbf{\thepage}}
\fancyhead[LO,RE]{\textbf{\leftmark}}
\fancyfoot[LE,RO]{Fabian M\"onkeberg}
\fancyfoot[LO,RE]{\today}
\renewcommand{\headrulewidth}{0.5pt}
\renewcommand{\footrulewidth}{0.5pt}

\begin{document}
\maketitle

%\tableofcontents

%\newpage

\section{Abstract}

This project presents an agent-based computational model of civil violence. Therein goes the central authority for it to suppress decentralized  rebellion. In general we build on an existing model of J.M.Epstein and try to make it more realistic by changing and adding certain properties of the ''agent'' and ''cop''. The focus of our work lies on the dynamics of the system and not on the political or social order. We are going to compare the results of our model, with those of J.M.Epstein.

\end{document}